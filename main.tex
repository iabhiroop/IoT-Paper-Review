\documentclass{article}

% Language setting
% Replace `english' with e.g. `spanish' to change the document language
\usepackage[english]{babel}

% Set page size and margins
% Replace `letterpaper' with `a4paper' for UK/EU standard size
\usepackage[letterpaper,top=2cm,bottom=2cm,left=3cm,right=3cm,marginparwidth=1.75cm]{geometry}

% Useful packages
\usepackage{amsmath}
\usepackage{graphicx}
\usepackage[colorlinks=true, allcolors=blue]{hyperref}

\title{Internet of Things for Smart Cities - Paper Review}
\author{Abhiroop Ippagunta}

\begin{document}
\maketitle

\begin{abstract}
The research paper focuses on the architecture and technologies for an urban Internet of Things (IoT) system. The authors argue that building a general architecture for the IoT is a complex task, due to the large variety of devices, link layer technologies, and services that may be involved. The paper specifically focuses on urban IoT systems, which are designed to support the vision of Smart Cities by using advanced communication technologies to provide added-value services for the administration of the city and citizens.
The paper provides a comprehensive survey of the enabling technologies, protocols, and architecture for an urban IoT. It also presents and discusses the technical solutions and best-practice guidelines adopted in the Padova Smart City project, which is a proof-of-concept deployment of an IoT island in the city of Padova, Italy, performed in collaboration with the city municipality.
Overall, the authors aim to provide an understanding of the specific requirements and challenges of an urban IoT system and the technical solutions that have been proposed and tested to address these challenges. It aims to inspire researchers, practitioners, and policymakers working in the field of IoT and Smart Cities.

\end{abstract}

\section{A take on use of IoT for a smart city}
The research paper explains the concept of Internet of Things which envisions a future where everyday objects are equipped with microcontrollers, transceivers for digital communication and suitable protocol stacks, making them able to communicate with one another and with users, becoming an integral part of the internet. 
The IoT concept aims at making the internet more immersive and pervasive by enabling easy access and interaction with a wide variety of devices such as home appliances, surveillance cameras, monitoring sensors, actuators, displays, vehicles and more, fostering the development of new applications that make use of the potentially enormous amount and variety of data generated by these objects to provide new services to citizens, companies and public administrations.
 An urban IoT, indeed, may bring a number of benefits in the management and optimization of traditional public services,
such as transport and parking, lighting, surveillance and maintenance of public areas, preservation of cultural heritage, garbage collection, salubrity of hospitals, and school.
 The aim of a smart city is to make better use of public resources, increasing the quality of services offered to citizens while reducing the operational costs of public administration by deploying an urban IoT, which is a communication infrastructure that provides unified, simple, and economical access to a plethora of public services.
 
 \section{Overall views}
\subsection{Agreements}
The IoT paradigm finds application in many different domains such as home automation, industrial automation, medical aids, mobile healthcare, elderly assistance, intelligent energy management and smart grids, automotive, traffic management and many others. However, this wide range of application scenarios makes it a challenge to identify solutions that can meet the requirements of all possible scenarios, leading to the proliferation of different and sometimes incompatible proposals for the practical realization of IoT systems.
The IoT concept, aims at making the Internet even more immersive and pervasive. Furthermore, by enabling easy access and interaction with a wide variety of devices such as, for instance, home appliances, surveillance cameras, monitoring sensors, actuators, displays, vehicles, and so on, the IoT will foster the development of several applications that make use of the potentially enormous amount and variety of data generated by such objects to provide new services to citizens, companies, and public administrations. This paradigm indeed finds application in many different domains, such as home automation, industrial automation, medical aids, mobile healthcare, elderly assistance, intelligent energy management and smart grids, automotive, traffic management, etc. The focus is on the application of the IoT paradigm in an urban context, which is of particular interest as it responds to the strong push of many national governments to adopt solutions in the management of public affairs, thus realizing the Smart City concept.
\subsection{Pitfalls}
The use of technology in smart cities can raise concerns about the collection, storage, and use of personal data, as well as the potential for cyber-attacks and breaches of security.
Smart cities rely heavily on technology, and not all citizens may have access to or be able to use this technology. This can lead to a digital divide and create a gap between those who have access to the benefits of smart city technology and those who do not.
The implementation of smart city technology can be costly, and cities may struggle to fund the necessary infrastructure and technologies.
Smart cities are heavily dependent on technology, if technology fails or is not available, it can disrupt the city's operations and services. There will also be a problem of limited flexibility and scalability.
\begin{figure}[h]
\centering
\includegraphics[width=0.9\textwidth]{iota-smart_city_components_mobile.jpg}
\caption{\label{fig:frog}Components in a smart city}
\end{figure}

\section{IoT for a Smart City}

\subsection{Main focus of IoT applications}

\begin{enumerate}
\item Structural Health of Buildings
\item Waste Management
\item Air Quality
\item Noise Monitoring
\item Traffic Congestion
\item City Energy Consumption
\item Smart Parking
\item Smart Lighting
\item Automation and Salubrity of Public Buildings
\end{enumerate}

\section{Architecture}

Most Smart City services are based on a centralized architecture, where a dense and heterogeneous set of peripheral devices deployed over the urban area generate different types of data that are then delivered through suitable communication technologies to a control center, where data storage and processing are performed.
A primary characteristic of an urban IoT infrastructure is its capability of integrating different technologies with the existing communication infrastructures in order to support a progressive evolution of the IoT, with the interconnection of other devices and to make the data collected by the urban IoT easily accessible by authorities and citizens, to increase the responsiveness of authorities to city problems and to promote the awareness and participation of citizens in public matters.
The different components of an urban IoT system, such as the web service approach for the design of IoT services, which requires the deployment of suitable protocol layers in the different elements of the network. They also briefly overview the link layer technologies that can be used to interconnect the different parts of the IoT and describe the heterogeneous set of devices that concur to the realization of an urban IoT.

\subsection{Web Service Approach for IoT Service Architecture}

Based on IETF standards because they are open and royalty-free, are based on Internet best practices, and can count on a wide community.
It proposes a reference protocol architecture that includes both an unconstrained and a constrained protocol stack. The unconstrained stack includes protocols that are commonly used in internet communications, such as XML, HTTP, and IPv4, while the constrained stack includes low-complexity protocols, the Efficient XML Interchange (EXI), the Constrained Application Protocol (CoAP), and 6LoWPAN, that are suitable for very constrained devices. This approach guarantees easy access and operation of IoT nodes with the internet.

In the protocol architecture, we can distinguish three distinct functional layers, namely 
\begin{enumerate}
\item Data- The eXtensible Markup Language (XML) is a common format for data representation in IoT, but its text-based nature and large size can be a problem for devices with limited capacity. The W3C has proposed the EXI format as a solution, which is a binary format that is compatible with XML. EXI has two types of encoding: schema-less and schema-informed. Schema-less encoding can be decoded by any EXI entity without any prior knowledge about the data, while schema-informed encoding assumes that the two EXI processors share an XML Schema.
\begin{figure}[h]
\includegraphics[width=0.5\textwidth]{arch.png}
\end{figure}
\item Application/ Transport- Most internet traffic is carried at the application layer by HTTP over TCP, but this protocol is not well-suited for deployment on constrained IoT devices due to its verbosity and complexity. The Constrained Application Protocol (CoAP) addresses these issues by proposing a binary format transported over UDP, which reduces the amount of correlated and redundant data. CoAP can easily interoperate with HTTP because it supports the ReST methods of HTTP. To enable interoperability between IoT devices and regular internet hosts, an HTTP-CoAP intermediary, can be deployed to translate requests and responses between the two protocols, allowing for transparent communication between native HTTP devices and applications.
\item Network- IPv4 is the leading addressing technology supported by Internet hosts, but the Internet Assigned Numbers Authority (IANA) has announced the exhaustion of IPv4 address blocks. IPv6 introduces overheads that are not compatible with the scarce capabilities of constrained nodes. The 6LoWPAN compression format for IPv6 and UDP headers can be used to overcome this problem and make IPv6 more suitable for low-power constrained networks. This approach raises a scalability problem and requires that the connection be initiated by the IPv6 nodes. A particular type of HTTP-CoAP cross proxy, the reverse cross
proxy behaves as being the final web server to the HTTP/IPv4 client and as the original client to the CoAP/IPv6 web server. IPv4/IPv6 conversion is internally resolved by the applied URI mapping function.
\end{enumerate}



\subsection{Link Layer Technologies}

An urban IoT system requires a set of link layer technologies that can easily cover a wide geographical area and support a large amount of traffic resulting from the aggregation of a high number of smaller data flows. The unconstrained technologies include traditional LAN, MAN, and WAN communication technologies such as Ethernet, WiFi, fiber optic, broadband Power Line Communication (PLC), and cellular technologies such as UMTS and LTE. These technologies are generally characterized by high reliability, low latency, and high transfer rates but are not suitable for peripheral IoT nodes due to their complexity and energy consumption. The constrained physical and link layer technologies are characterized by low energy consumption and relatively low transfer rates, typically smaller than 1 Mbit/s. These links usually exhibit long latencies due to the low transmission rate at the physical layer and the power-saving policies implemented by the nodes to save energy, which usually involve duty cycling with short active periods.

\subsection{Devices in the Architecture}

Backend Servers that contain:
\begin{enumerate}
\item Database management systems
\item Web sites
\item Enterprise resource planning systems (ERP)
\end{enumerate}

Gateways whose role is to interconnect the end devices to the main communication infrastructure of the system. Gateway devices shall also provide the interconnection between unconstrained link layer technologies, mainly used in the core of the IoT network, and constrained technologies that, instead, provide connectivity among the IoT peripheral nodes.

IoT Peripheral Nodes are the devices in charge of producing the data to be delivered to the control center, which are usually called IoT peripheral nodes or, more simply, IoT nodes. Generall speaking, the cost of these devices is very low.
\begin{figure}
\includegraphics[width=0.7\textwidth]{arch_main.png}
\end{figure}
\section{Practical results for smart city}
The Padova Smart City system is an example of a system that collects data from various sensors placed throughout a city. The data collected includes temperature, humidity, light, and benzene readings over a period of 7 days. The data is presented in the form of plots that show the actual readings as thin lines and the readings after being processed by a moving average filter as thick lines. The light measurements show a regular pattern corresponding to day and night periods. During the day, the measure reaches the saturation value, while during nighttime, the values are more irregular due to reflections produced by vehicle lights. A similar pattern is exhibited by the humidity and temperature measurements, however, they are much more noisy than those for light. The benzene measurements also reveal a decrease of the benzene levels at nighttime, as expected due to the lighter night traffic, but surprisingly there is no evident variations in the daytime benzene levels during the weekend. A peak of benzene measured in the early afternoon of October 29 can be observed, and by examining the readings of the other sensors in the same time interval, it can be noted that this peak is likely due to a quick rainstorm temporarily obscuring the sunlight, producing congestion in the road traffic, and in turn, a peak of benzene in the air.

\end{document}